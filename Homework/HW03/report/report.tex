\documentclass{article}
\usepackage[UTF8]{ctex}
\usepackage{graphicx}
\usepackage{amsmath}
\usepackage{booktabs}
\usepackage{subcaption}
\usepackage{geometry}
\geometry{a4paper, left=2.5cm, right=2.5cm, top=2.5cm, bottom=2.5cm}

\title{作业3实验报告: 基于神经网络的昆虫分类}
\author{21 刘行}
\date{\today}

\begin{document}
\maketitle

\section{问题背景}
在生物分类任务中, 自动化分类模型需要根据样本特征 (如昆虫的体长和翼长) 预测其类别. 本实验使用神经网络对两类数据集 (常规数据与含噪声数据) 进行分类, 通过对比不同参数配置下的分类准确率, 分析模型的鲁棒性与影响因素.

\section{方法描述}
\subsection{模型结构}
采用全连接神经网络, 结构如下:
\begin{itemize}
    \item 输入层: 2个神经元 (对应体长和翼长) 
    \item 隐藏层: 默认配置为$[16, 32, 64]$, 激活函数可选ReLU/Tanh/Sigmoid
    \item 输出层: 3个神经元 (对应类别0/1/2) , 使用Softmax归一化
\end{itemize}

\subsection{损失函数与优化器}
\begin{equation}
    \mathcal{L} = \text{CrossEntropyLoss}, \quad \text{优化器: Adam (学习率 } 10^{-3}\text{)}
\end{equation}

\section{实验设置}
\subsection{数据集说明}
\begin{itemize}
    \item 数据集1: 常规, 训练集 (\texttt{insects-training.txt}) 与测试集 (\texttt{insects-testing.txt})
    \item 数据集2: 含噪声, 训练集 (\texttt{insects-2-training.txt}) 与测试集 (\texttt{insects-2-testing.txt})
    \item 测试集分为两部分: 前60个为训练集子集, 后150个为全新数据
\end{itemize}

\subsection{训练参数}
\begin{itemize}
    \item 训练轮次: 100
    \item 批大小: 32
    \item 隐藏层维度: $[16, 32, 64]$
    \item 激活函数: ReLU (默认) 
\end{itemize}

\section{实验结果}
\subsection{分类准确率对比}
\begin{table}[ht]
    \centering
    \caption{不同数据集下的测试准确率 (\%)}
    \label{tab:acc}
    \begin{tabular}{cccc}
        \toprule
        数据集 & 前 60 样本 & 后 150 样本 & 总体 \\
        \midrule
        数据集1 & 96.67 & 94.67 & 95.24 \\
        数据集2 & 90.00 & 92.00 & 91.43 \\
        \bottomrule
    \end{tabular}
\end{table}

\subsection{参数影响分析}
\begin{itemize}
    \item \textbf{隐藏层维度}: 增加层数 (如从$[16,32]$到$[16,32,64]$) 可提升复杂数据拟合能力, 但可能引发过拟合.
    \item \textbf{激活函数}: ReLU在两类数据集上表现最优, Tanh次之, Sigmoid因梯度消失问题导致收敛缓慢.
\end{itemize}

\subsection{关键发现}
\begin{itemize}
    \item 测试集分割意义: 前 60 样本验证模型对训练数据的记忆能力, 后150样本评估泛化性能. 实验表明模型未严重过拟合 (后150准确率高于85\%).
    \item 噪声影响: 数据集 2 中噪声使分类边界模糊, 导致后 150 样本准确率下降约 2.67\%.
    \item 噪声可能导致前 60 个数据受污染情况更严重, 使得数据集 2 中前 60 样本测试结果反而小与后 150 样本.
\end{itemize}

\section{结论}
本实验验证了全连接神经网络在昆虫分类任务中的有效性. 模型在常规数据上表现优异 (总体准确率95.24\%), 但对噪声敏感. 未来可通过数据增强或Dropout层提升鲁棒性.

\end{document}
